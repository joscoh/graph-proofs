\documentclass{article}
\usepackage{amsmath}
\usepackage{amsthm}
\usepackage[utf8]{inputenc}
\usepackage[english]{babel}
\usepackage{listings}
\usepackage{algpseudocode}
 
\newtheorem{theorem}{Theorem}[section]
\newtheorem{corollary}{Corollary}[theorem]
\newtheorem{lemma}[theorem]{Lemma}
\title{DFS Proofs}
\author{Josh Cohen}

\begin{document}
\maketitle
The document contains sketches of the major lemmas and theorems proved.
\section{The algorithm}
I have defined a DFS implementation both as an inductive relation and as an executable function (and have proven their equivalence). For the purposes of proofs, the inductive relation is used, as that makes it easier to reason about how the algorithm proceeds as well as to state theorems that start with "at the time when v is discovered ...". Throughout the algorithm, we keep track of the following:
\begin{itemize}
\item
$remain\_d$: The set of vertices that have not yet been discovered
\item
$remain\_f$: The set of vertices that have not yet been finished
\item
$d\_times$: A map of vertices to their discovery times
\item
$f\_times$: A map of vertices to their finish times
\item
$time$: The current time
\item
$f$: The current forest that we build up throughout the algorithm
\item
$stack$: The stack, which consists of tuples of type (vertex, option vertex), and which represent a vertex and its optional parent
\end{itemize}
For simplicity, as well as consistency with the CLRS algorithm, we define the following:
\begin{itemize}
\item
A vertex is white if it is in both $remain\_d$ and $remain\_f$ (This is equivalent to being in $remain\_d$ alone)
\item
A vertex is gray if it is $remain\_f$ but is not in $remain\_d$
\item
A vertex is black if it is not in $remain\_f$
\end{itemize}
The basic execution of the algorithm is as follows:
\pagebreak
\begin{algorithmic}
\While {$remain\_d$ is not empty}
\While{The vertex on the top of the stack is black}
\State{Pop the stack}
\EndWhile
\If{The stack is empty}
\State{Push the smallest white vertex on the stack with parent None}
\EndIf
\State{Let (v, o) be the element on the top of the stack}
\If{v is white and $o$ is None}
\State{Color v gray}
\State{Add v with the current time to $d\_times$}
\State{Add v as a new root in the DFS forest}
\State{Add all of v's white neighbors to the stack with parent (Some v)}
\ElsIf{v is white and o=Some p}
\State{Color v gray}
\State{Add v with the current time to $d\_times$}
\State{Add v as a child of p in the DFS forest}
\State{Add all of v's white neighbors to the stack with parent (Some v)}
\ElsIf{v is gray}
\State{Color v black}
\State{Add v with the current time to $f\_times$}
\State{Pop the stack}
\EndIf
\State{Increment the current time}
\EndWhile
\end{algorithmic}
The intent of this algorithm is to adhere as closely as possible to the CLRS implementation while replacing the recursive nature of the algorithm with an explicit stack. The one key difference is that before we look at the vertex on the top of the stack in detail, we remove all black vertices from the top of the stack, and then add the next white vertex to the stack. The ensures that in each iteration of this algorithm, we are either discovering or finishing a vertex, which makes the proof of termination much simpler.
\section{Parentheses Theorem}
\begin{lemma}
If (u, Some v) is on the stack, then v is either gray or black.
\end{lemma}
\begin{proof}
We prove the claim by induction on the DFS inductive relation. This gives us 3 cases:
\begin{enumerate}
\item
If the top of the stack is (x, None) and x was white, then the only vertices with a parent on the stack have x as a parent, and x is gray.
\item
If the top of the stack is (x, Some y) and x was white, then the stack is $l1 ++ (x, \text{Some }y) l2$, where l1 consists of x's children and the previous stack included (x, Some y) ++ l2. So if (u, Some v) is in l1, then v = x, which is gray. If it is in the rest of the stack, the claim follows by induction (since v was gray or black before, and we did not change it).
\item
If the top of the stack is (x, o) and x is gray, then the claim follows immediately by induction.
\end{enumerate}
\end{proof}
The following lemma characterizes the stack when a single vertex is gray. This holds for any such state, and is useful for then proving claims about specific states within this time interval.
\begin{lemma}
If v is gray, then the stack looks like
\begin{lstlisting}[language=Haskell]
l1 ++ (v, o) :: l2
\end{lstlisting}
And the following hold:
\begin{itemize}
\item
v does not appear in l1 (that is, for all o, (v, o) does not appear in l1)
\item
v does not appear as a parent in l2
\item
For all (x, Some y) in l1, either y=v or y is a descendant of v in the DFS forest
\end{itemize}
\end{lemma}
\begin{proof}
We will prove the claim by induction on the DFS inductive relation. This gives us 3 cases:
\begin{enumerate}
\item
If the top of the stack is (x, None) and x was white, then it must be the case that v = x (since there are no other gray vertices on the stack). Then l2 = nil, so the second condition holds trivially. l1 consists of v's children who are white, which does not include v (since v is gray now), and for every (x, Some y) in l1, y = v, so the third condition is true.
\item
If the top of the stack is (x, Some y) and x was white, then either v = x or v $\neq$ x. If v = x, then l1 consists only of v's children who are white, so by the same logic as the previous part, conditions 1 and 3 hold. Suppose condition 2 did not hold, so (u, Some x) was in l1. Then, by Lemma 2.1, x was gray in the previous state, a contradiction.
\\If v $\neq$ x, then v was gray in the previous state as well. So the previous stack looked like $l ++ (v, o) :: t$, and once we popped off the black vertices, the current stack looks like $l'' ++ (x, \text{Some }y) ++ l' ++ (v, o) :: t$ (since we could not have popped (v,o) as it is gray), and $l''$ consists only of children of x. So let $l1=l'' ++ (x, \text{Some }y) ++ l'$ and let $l2 = t$.  $v$ does not appear in $l'$ by induction. $v \neq x$ by assumption, and v cannot appear in $l1$ since it is gray. Thus, condition 1 is true. Condition 2 follows by induction. Condition 3 holds on $(x, Some y) ++ l'$ by induction, so we need to show that for all $(a, Some b)$ in $l1$, either $b=v$ or $v$ is a descendant of $v$. Since $l1$ consists of children of $x$, $b=x$, and by induction, y = v or y is a descendant of v. Then, we add x as a child of y in this step, so either x is a child of v, or x is a child of y, who is a descendant of v. So x is a descendant of v, proving condition 3.
\item
If the top of the stack is (x, Some y) and x is gray, then the claim follows by induction (since (v, o) $\neq$ (x, Some y) as x turns black while v is gray).
\end{enumerate}
\end{proof}
In particular, we can expand on the previous result to get that the left side of the stack is constant when a vertex is gray. We only add and remove from the right side of the stack.
\begin{lemma}
If v is gray and the stack looks like $l1 ++ (v, o) :: l2$ such that v does not appear in l1, then at any future point when v is gray, the stack looks like $l' ++ (v, o) :: l2$ and $v$ does not appear in $l'$. In other words, the left side of the stack is constant.
\end{lemma}
\begin{proof}
It suffices to prove this for 1 step of the algorithm. We proceed by inversion on the inductive DFS relation:
\begin{enumerate}
\item
If the top of the stack is (x, None) and x was white in the previous step, then the only gray vertex is x. But x was not gray in the previous step, a contradiction.
\item
If the top of stack is (x, Some y) and x was white in the previous step, then the stack now looks like $c ++ (x, \text{Some }y) ++ l' ++ (v, o) :: l2$, where $c$ are the white children of $x$, $l'$ is a sublist of $l1$, and $l2$ is the same as in the previous step. We must show that $v$ does not appear in $c ++ (x, \text{Some }y) ++ l'$. $v$ is gray, so it is not in $c$. $v \neq x$ since $x$ was white in the previous step. $v$ is not in $l'$ by assumption.
\item
If the top of the stack is (x, o) and x was gray in the previous step, then either v = x or $v\neq x$. If v = x, then v is no longer gray in the current state, a contradiction. Otherwise, $v \neq x$ so the claim follows by assumption (we only got rid of a vertex in $l1$).
\end{enumerate}
\end{proof}
The following lemmas are mainly proved by tedious reasoning about the stack, using the above two lemmas and equating various representations of the stack. For now, I am omitting the proofs.
\begin{lemma}
If the stack looks like
\begin{lstlisting}[language=Haskell]
l1 ++ (u, Some p) :: l2 ++ (v, o) :: l3
\end{lstlisting}
and v does not occur in (l1 ++ (u, Some p) :: l2),
if $u$ and $v$ are gray, then after 1 step, if $v$ is black, then $u$ is black.
\end{lemma}
\begin{lemma}
If two gray vertices u and v are on the stack such that u occurs before v in the stack (ie, the stack looks like 
\begin{lstlisting}[language=Haskell]
l1 ++ (u, Some p) :: l2 ++ (v, o) :: l3
\end{lstlisting}
and v does not occur in (l1 ++ (u, Some p) :: l2), then if after one step, u is still gray, then there is an l1 such that the stack looks like
\begin{lstlisting}[language=Haskell]
l1 ++ (u, Some p) :: l2 ++ (v, o) :: l3
\end{lstlisting}
and v does not occur in (l1 ++ (u, Some p) :: l2).
\end{lemma}
\begin{lemma}
If the stack looks like
\begin{lstlisting}[language=Haskell]
l1 ++ (u, Some p) :: l2 ++ (v, o) :: l3
\end{lstlisting}
and v does not occur in (l1 ++ (u, Some p) :: l2),
if $u$ and $v$ are gray, then at any future step, if $v$ is black, then $u$ is black.
\end{lemma}
\begin{proof}
Follows from the previous two lemmas by induction.
\end{proof}
We can now begin reasoning about discovery and finish times more directly.
\begin{lemma}
At the discovery time of v, the following hold:
\begin{itemize}
\item
v is gray
\item
The stack looks like 
\begin{lstlisting}[language=Haskell]
l1 ++ (v, o) :: l2
\end{lstlisting}
\item
v does not occur in l1
\item
Every vertex in l1 is white
\end{itemize}
\end{lemma}
\begin{proof}
This follows easily from the inductive relation. It only requires knowing that any vertex's finish time must be $\leq$ the time of the current step, which is a very simple proof.
\end{proof}
\begin{lemma}
If p and v are two distinct vertices such that p is gray when v is discovered, then at any future point, if p is black, so is v.
\end{lemma}
\begin{proof}
Since p is gray at time dv, by Lemma 2.7, we know that p is in l2. Thus, we can apply Lemma 2.6 for the desired result.
\end{proof}
\begin{lemma}
If a vertex u has a finish (or discovery) time, then there was a previous step at which it was finished (or discovered).
\end{lemma}
\begin{proof}
Follows from induction on the DFS step relation.
\end{proof}
\begin{lemma}
If v is a vertex in the graph that is not black, then there is a future state during which it will be finished.
\end{lemma}
\begin{proof}
We know that every vertex is finished at the end. By Lemma 2.9, there is a state during which it was finished. If that state occured before the current one, then v is black, a contradiction. So it must occur after.
\end{proof}
Last two key lemmas for Parentheses Theorem:
\begin{lemma}
If u and v are two distinct vertices such that u is gray when v is discovered, then v finishes before u.
\end{lemma}
\begin{proof}
By Lemma 2.8, at any future point, if u is black, then so is v. So at the time when u becomes black, v must have already finished. Thus v finishes before u.
\end{proof}
\begin{lemma}
For any distinct vertices u and v, if $ du < dv < fu$, then
\\ $du < dv < fv < fu$.
\end{lemma}
\begin{proof}
At the time when v is discovered, u must already be discovered, but it is not finished. So u is gray. Thus, the claim follows from Lemma 2.11.
\end{proof}
\begin{theorem}[Parentheses Theorem]
For any distinct vertices u and v, exactly one of the following holds:
\begin{itemize}
\item
$du < dv < fv < fu$
\item
$dv < du < fu < fv$
\item
$du < fu < dv < fv$
\item
$dv < fv < du < fu$
\end{itemize}
\begin{proof}
For any two vertices u and v, either $du < dv$ or $dv < du$. Without loss of generality, suppose $du < dv$. Then, since every vertex is discovered before it finishes, we have either $du < fu < dv$, in which case we have that $du < fu < dv < fv$, or $du < dv < fu$, in which case we apply Lemma 2.12 for the desired result. The other case follows similarly.
\end{proof}
\end{theorem}
\section{Corollary 22.8 (in CLRS)}
\begin{lemma}
If u and v are distinct vertices, then if u is gray when v is discovered, v is a descendant of u in the DFS forest.
\end{lemma}
\begin{proof}
When v is discovered, either it appears on the stack as (v, None), or as (v, Some p). In the first case, v is the only gray vertex on the stack, so we have a contradiction. In the second case, the stack looks like $c ++ (v, \text{Some }p) :: l$ where $c$ are the children of $v$ (all white) and $u$ appears in $l$ (since it is gray). By Lemma 2.2, we know that the stack also looks like $l1 + (u, o) :: l2$, where $u$ does not appear in $l1$ and for all (a, Some b) in $l1$, either $b=u$ or $b$ is a descendant of $u$ in the DFS forest. Clearly, (v, Some p) must occur in $l1$ (otherwise this implies that $u$ is a child of $v$ and is white, a contradiction). Therefore, either $p=u$ or $p$ is a descendant of $u$, so since we add $v$ as a child of $p$, $v$ is a descendant of $u$.
\end{proof}
\begin{lemma}
For any two distinct vertices u and v, if $du < dv < fv < fu$, then v is a descendant of u in the DFS forest.
\end{lemma}
\begin{proof}
Since $du < dv < fu$, at time dv, u has started but not finished, so it is gray. Thus, the result follows from Lemma 3.1.
\end{proof}
\begin{lemma}
If (u, Some v) appears on the stack, then v is gray.
\end{lemma}
\begin{proof}
We proceed by induction on the DFS inductive relation.
\begin{enumerate}
\item
If (x, None) was on the top of the stack and x was white, then (u, Some v) must occur in the portion of the stack that consists only of children of x, so v = x. x is gray, so the claim is true.
\item
If (x, Some y) was on the top of the stack and x was white, then if (u, Some y) occurs in the portion of the stack consisting of children of x, the claim is true for the same reason. Otherwise, everything else was on the stack previously, so the claim follows by induction.
\item
If (x, o) was on top of the stack and x was gray, then it suffices to show that v $\neq$ x (everything else that was gray is still gray). Assume $v=x$. Then $x$ was gray in the previous state, so by lemma 2.2, the previous stack was $l1 ++ (x, o') :: l2$ and $x$ does not appear as a parent in l2 and $x$ does not appear in $l1$. After we remove the black vertices from the top of the stack, the current stack is $l ++ (x, o') :: l2$. But the top of the stack is $(x, o)$, so $l=[]$ (if not, then $x$ appears in $l1$, a contradiction). So we know that $(u, Some v)$ cannot occur in l2. If (u, Some v) = (x, o'), then $u=v=x$, a contradiction, since $(x, Some x)$ cannot appear on the stack.
\end{enumerate}
\end{proof}
\begin{lemma}
If (u, v) is an edge in the DFS forest (ie: v is a child of u in the DFS forest) at time dv, then u is gray at time dv.
\end{lemma}
\begin{proof}
When v is discovered, v is not already in forest, and we add an edge from its parent in the stack (u). Therefore, we must only prove that u is gray, which follows from Lemma 3.3 
\end{proof}
\begin{lemma}
(u,v) is an edge in the DFS forest iff it is an edge at time dv.
\end{lemma}
\begin{proof}
Follows from inspection of the algorithm: we only add an edge for a given vertex when it becomes gray, which can happen only once.
\end{proof}
\begin{lemma}
If (u,v) is an edge in the DFS forest, then $du < dv < fv < fu$.
\end{lemma}
\begin{proof}
Since (u,v) is in the DFS forest, by Lemma 3.5, it is an edge at time dv. By lemma 3.4, u is gray at time dv. By lemma 2.11, v finishes before u, and by Lemma 2.12, $du < dv < fv < fu$.
\end{proof}
\begin{lemma}
Every vertex in the forest was in the original graph.
\end{lemma}
\begin{lemma}
At the end of the algorithm, every vertex in the original graph is in the forest.
\end{lemma}
\begin{proof}
We begin the algorithm with all the vertices in the graph in $remain\_d$. A vertex is added to the forest when it becomes gray (or alternatively, gets a discovery time). A vertex has a discovery time iff it is not white. At the end of the algorithm, there are no white vertices. Thus, all vertices have discovery times, and so they were all added to the forest.
\end{proof}
\begin{lemma}
If u is a descendant of v in the DFS forest, then $u \neq v$ (This is equivalent to proving that the DFS forest is acyclic).
\end{lemma}
\begin{proof}
Follows from properties of forests: If u is a descendant of v, then u had to exist in the forest before v was added, but v could not have existed in the forest, so u and v cannot be the same vertex.
\end{proof}
\begin{lemma}
If u and v are two distinct vertices such that v is a descendant of u in the DFS forest, then $du < dv < fv < fu$.
\end{lemma}
\begin{proof}
Follows by induction on the proof that v is a descendant of u. If v is a child of u, it follows by Lemma 3.6. Otherwise, if w is a descendant of u, and v is a child of w, by the induction hypothesis, $du < dw < fw < fu$, and by Lemma 3.6, $dw < dv < fv < fw$. The claim follows from these inequalities.
\end{proof}
\begin{theorem}[Corollary 22.8]
For any two vertices u and v, v is a descendant of u in the DFS forest iff $du < dv < fv < fu$.
\end{theorem}
\begin{proof}
$(\implies)$: since v is a descendant of u, $v\neq u$ by Lemma 3.9. The result follows by Lemma 3.10.
\\$(\impliedby)$: Since u and v have different discovery times, they cannot be the same vertex. Thus $v\neq u$. The result follows by Lemma 3.2.
\end{proof}
\section{White Path Theorem}
\begin{lemma}
If (u, Some v) is on the stack, then the edge (v, u) exists in the graph.
\end{lemma}
\begin{proof}
The only time it could have been added is when v was discovered, when we add v's children to the stack.
\end{proof}
\begin{lemma}
If v is a child of u in the DFS forest, then (u,v) exists in the graph.
\end{lemma}
\begin{proof}
The only time we add the child is when we discover v and see (v, Some u) on the stack. Thus, the claim follows by Lemma 4.1.
\end{proof}
\begin{lemma}
If $v_0, v_1,...,v_k$ forms a path from $u$ to $v$ in the DFS forest, then $v_0,v_1,...v_k$ forms a path from $u$ to $v$ in the original graph.
\end{lemma}
\begin{proof}
Follows easily by induction on the path and Lemma 4.2.
\end{proof}
\begin{lemma}
If v is a descendant of u in the DFS forest at the end of the algorithm, then v was white at time du.
\end{lemma}
\begin{proof}
By Theorem 3.11, $du < dv < fv < fu$. Therefore, v has not yet been discovered at time du, so it is white.
\end{proof}
\begin{lemma}
If v is a descendant of u in the DFS forest, then at time du, there exists a path of vertices from u to v such that every vertex on the path (except for u) are white.
\end{lemma}
\begin{proof}
If v is a descendant of u, then there is a path of vertices $v_0,...v_k$ such that for all $v_i$, $v_i$ is a descendant of u. Thus, let this be the path. We want to show that all vertices along this path (as well as v) are white. For any given $v_i$ (as well as v), this follows from Lemma 4.4.
\end{proof}
\begin{lemma}
If (u, Some p) is on the stack, then it is on the stack at time du.
\end{lemma}
\begin{proof}
We want to prove the claim for all states, so consider 3 possible cases:
\begin{enumerate}
\item
The state is the state at which u is discovered. Then, the claim follows by assumption.
\item
The state occurs before u was discovered. Then, we only ever remove a vertex from the stack if it is gray or black, but u is white. So u cannot be removed from the stack during any one of these states. Thus, it must still be there at time du.
\item
The state occurs after u was discovered. Then, since we never add u to the stack again once it is no longer white, it must have been on the stack during the previous state. Thus, it cannot have been added at any point after u was discovered.
\end{enumerate}
\end{proof}
\begin{lemma}
If (u, Some p) is on the stack, then p is gray when u is discovered.
\end{lemma}
\begin{proof}
By lemma 4.6, (u, Some p) is on the stack when u is discovered. By Lemma 3.3, p is gray at this time.
\end{proof}
\begin{lemma}
If (u, Some p) appears on the stack at any point, then u is a descendant of p in the DFS forest when the algorithm finishes.
\end{lemma}
\begin{proof}
If (u, Some p) appears on the stack, then p must be gray at time du by Lemma 4.7. Therefore, by Lemma 2.11, u finishes before p, so we have that $dp < du < fp$. Then, by lemma 2.12, $dp < du < fu < fp$, and by Theorem 3.11, this implies that $u$ is a descendant of $p$ in the DFS forest.
\end{proof}
\begin{lemma}
If the edge (p, u) exists in the graph and if u is white at time dp, then u is a descendant of p in the DFS forest (when the algorithm finishes).
\end{lemma}
\begin{proof}
At time dp, since u is white, u will be added to the stack as a child of p (ie, (u, Some p) will be added to the stack). Thus, by Lemma 4.8, u is a descendant of p in the DFS forest when the algorithm terminates.
\end{proof}
\begin{lemma}
If there is a white path from u to v at time du, then v is a descendant of u in the DFS forest.
\end{lemma}
\begin{proof}
If the path is empty (implying that (u,v) exists in the graph), then the claim follows from Lemma 4.9. Otherwise, the path is non-empty. So there is a vertex w such that there is a white path from u to w, w is white at time du, (w,v) exists in the graph, and v is white at time du. We consider two cases:
\begin{enumerate}
\item
v is white at time dw. Then, by Lemma 4.9, v is a descendant of w, and since w is a descendant of u, we know that v is a descendant of u, proving the claim.
\item
v is not white at time dw. Then, since w is a descendant of u, by Theorem 3.11, $du < dw < fw < fu$. Thus, since $v$ is not white at dw, it have been discovered between $du$ and $dw$. Therefore, we have that $du < dv < dw < fw < fu$, or in particular, $du < dv < fu$. Thus, by lemma 2.12, $du < dv < fv < fu$, and by Theorem 3.11, this implies that $v$ is a descendant of $u$ in the DFS forest, which was what we wanted to show.
\end{enumerate}
\end{proof}
\begin{theorem}[White Path Theorem]
When the algorithm finishes, v is a descendant of u in the DFS forest iff at time du, there is a path consisting of white vertices from u to v in the original graph.
\end{theorem}
\begin{proof}
Follows from Lemma 4.10 and Lemma 4.5.
\end{proof}
\section{DFS Specification}
The following defines a set of properties that define a DFS algorithm that computes start and finish times. The above DFS algorithm satisfies this specification, but separating out the properties from the implementation means that we can prove the correctness of several applications of DFS no matter underlying implementation is used.
\begin{enumerate}
\item
There is a function dfs that takes in a vertex option (an optional starting vertex) and a graph and produces a forest, a function from vertices to nats (discovery times), and a function from vertices to nats (finish times).
\item
There is a type called state parameterized by a graph and a vertex option. It is used in the function time\_of\_state, which takes in a state and produces a nat. This allows us to say "at the time when vertex v was discovered" in the white path theorem.
\item
For any vertex, if it is discovered at time $n$, then there is a state with time $n$.
\item
We define a white vertex in a given state to be one where the state's time is less than the vertex's discovery time, gray for a state time in between the discovery and finish times, and black after the finish time.
\item
Discovery and finish times are unique among vertices, and no vertex is discovered at the same time another finishes.
\item
State times are unique (ie, there is only one state with each time).
\item
The parentheses theorem, Corollary 22.8, and the white path theorem all hold.
\item
The dfs forest and the original graph have the same vertices.
\item
All edges in the dfs forest are in the original graph.
\item
If we start at a given vertex $v$ (ie, the option we pass into dfs is $Some v$), and $v$ is in the graph, then $v$ has a smaller discovery time than any other vertex in the graph.
\end{enumerate}
\section{Applications of DFS}
Now we can prove several applications, algorithms, and additional properties of DFS based only on the above specification.
\begin{lemma}
If $u$ is a descendant of $v$ in the DFS forest, then there is a path from $v$ to $u$ in the original graph.
\end{lemma}
\begin{proof}
Follows from induction and property 9.
\end{proof}
\begin{theorem}[Reachability]
For any two distinct vertices $u$ and $v$, there is a path from $u$ to $v$ in the original graph iff $v$ is a descendant of $u$ in the DFS forest when we begin DFS with $u$.
\end{theorem}
\begin{proof}
For the forward direction, if there is a path from $u$ to $v$, then since $u$ has the smallest discovery time of any vertex in the graph (by Property 10), the path from $u$ to $v$ must be white at time $du$. Thus, by the white path theorem, $v$ is a descendant of $u$. Lemma 6.1 proves the reverse direction.
\end{proof}
\begin{corollary}
The existence of a path between two vertices is decidable.
\end{corollary}
\begin{proof}
By Theorem 6.2, there is a path between $u$ and $v$ iff $v$ is a descendant of $u$ in the DFS forest when we start with $u$. By Corollary 22.8, this is equivalent to the condition that $du < dv < fu < fv$, which is decidable.
\end{proof}
A less trivial application of DFS is cycle detection. For this, we need the following results about paths and cycles:
\begin{lemma}
If there is a path $p=u,x_1,x_2,...x_n,v$ from $u$ to $v$ such that $u=x_i$ for some $i$, then $u,x_{i+1},...x_n,v$ is a path from $u$ to $v$ that is shorter than $p$. 
\end{lemma}
\begin{lemma}
If there is a path $p=u,x_1,...x_{i-1},w,x_{i+1},...x_{j-1},w,x_{j+1},...x_n,v$, then we also have the path $p'=u,x_1,x_{i-1},w,x_{j+1},...,x_n,v$ which is strictly shorter than the original path. In other words, if there is a cycle along the path, then we can remove the cycle to get a smaller path.
\end{lemma}
Define a nontrivial cycle to be one that contains at least 2 vertices. In other words, a cycle that is not only a self-loop or consists of a single vertex again and again.
\begin{lemma}
If there is a nontrivial cycle starting and ending at $u$, then there is a cycle with no duplicate elements from $u$ to $u$ that contains all the vertices of the original cycle.
\end{lemma}
\begin{proof}
We proceed by strong induction on the length of the cycle. So assume that if there is a nontrivial cycle from $u$ to $u$ of length at most $n$, then we can find a cycle with no duplicates and the same vertices. We want to show this is true if there is a cycle of length $n+1$. If this cycle has no duplicates, then we are good. Otherwise, there is a vertex that is repeated in the cycle. If it is $u$, then by lemma 6.3, we can form a smaller cycle and apply the induction hypothesis. If it is a different vertex $w$, then we can use Lemma 6.3 to find a smaller cycle and again apply the induction hypothesis. Thus, the claim is true.
\end{proof}
Finally, we will need the following useful lemma, which provides yet another useful way of classifying descendants:
\begin{lemma}
$v$ is a descendant of $u$ in the DFS forest iff $u$ is gray when $v$ is discovered.
\end{lemma}
\begin{proof}
Assume that $v$ is a descendant of $u$ in the DFS forest. Then, by Corollary 22.8, $du < dv < fv < fu$. At time $dv$, therefore, $u$ has been discovered but not finished, so it is gray.
\\Now assume that $u$ is gray when $v$ is discovered. This means that $du < dv < fu$. By the parentheses theorem, it must be the case that $du < dv < fv < fu$. By Corollary 22.8, $v$ is a descendant of $u$.
\end{proof}
Now we can prove the results about cycle detection and back edges. We define a back edge to be an edge from $u$ to $v$ in the original graph, such that $u$ is a descendant of $v$ in the DFS forest. We also have the following equivalent characterization:
\begin{lemma}
$(u,v)$ is a back edge iff $(u,v)$ exists in the graph and $v$ is gray when $u$ is discovered.
\end{lemma}
\begin{proof}
Follows from Lemma 6.6.
\end{proof}
Now, we can provide a criterion for (nontrivial) cycle detection
\begin{theorem}[Cycle Detection]
There is a nontrivial cycle in a graph $g$ iff $g$ has a back edge.
\end{theorem}
\begin{proof}
First, assume that there is a back edge $(u,v)$. Then, this means that $u$ is a descendant of $v$ in the DFS forest. By lemma 6.1, there is a path from $v$ to $u$. Since we also have the edge $(u,v)$, this forms a cycle.
\\Now, assume that there is a nontrivial cycle. By lemma 6.5, let this cycle have no duplicates. It is still nontrivial. Then, let $x$ be the first vertex discovered in the cycle. Consider the vertex before $x$ in the cycle (which must be different from $x$ because the cycle has no duplicates). Call this vertex $y$. At time $dx$, every other vertex on the cycle is white, so this forms a white path from $x$ to $y$. Thus, by the white path theorem, $y$ is a descendant of $x$. Then, since $y$ is a descendant of $x$, by Lemma 6.6, $x$ is gray at time $dy$. Therefore, by Lemma 6.7, there is a back edge.
\end{proof}
Finally, we want to prove that DFS can be used to easily and efficiently implement topological sorting. We will say that a list of vertices $l$ represents a topological sorting of a graph $g$ if the following 3 conditions apply:
\begin{enumerate}
\item
$l$ contains every vertex in $g$.
\item
$l$ contains no duplicates.
\item
For any $u$ and $v$ such that $l = l1 ++ u :: l2 ++ v :: l3$, there is no edge from $v$ to $u$ in the original graph.
\end{enumerate}
We will need a single helper lemma:
\begin{lemma}
If $g$ is acyclic (we are working only with directed graphs, so this is equivalent to $g$ being a DAG), then for any edge $(u,v)$ in $g$, $fv < fu$. 
\end{lemma}
\begin{proof}
At time $du$, $v$ is either white, gray, or black. If $v$ is white, then it is a descendant of $u$ in the DFS forest (by the white path theorem), so by Corollary 22.8, $fv < fu$. If $v$ is gray, then we have a back edge, which contradicts the fact that $g$ is acyclic (and thus contains no nontrivial cycles). If $v$ is black, then it has already finished, and $u$ has not yet finished, so $fv < fu$.
\end{proof}
Finally, we can provide a simple condition for topological sorting: a list sorted in reverse order of finish times.
\begin{theorem}[Topological Sorting]
If $g$ is acyclic and $l$ is a list containing all the vertices of $g$ that is sorted in reverse order of finish times, then $l$ is a topological sorting of $g$.
\end{theorem}
\begin{proof}
Clearly the first two conditions are met by assumption. We must show that if $l = l1 ++ u :: l2 ++ v :: l3$, then $(v,u)$ does not exist. Suppose it does exist. Then, by lemma 6.9, $fu < fv$. But then $v$ should appear later in $l$, since it is sorted. This is a contradiction. Thus, we have a topological sorting.
\end{proof}
\end{document}