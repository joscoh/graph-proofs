\documentclass{article}
\usepackage{amsmath}
\usepackage{amsthm}
\usepackage[utf8]{inputenc}
\usepackage[english]{babel}
\usepackage{listings}
 
\newtheorem{theorem}{Theorem}[section]
\newtheorem{corollary}{Corollary}[theorem]
\newtheorem{lemma}[theorem]{Lemma}
\title{DFS Proofs}
\author{Josh Cohen}

\begin{document}
\maketitle
\section{Parenthesis Theorem}
\begin{lemma}
If $(x, \text{Some } y)$ is on the stack, then the stack looks like
\begin{lstlisting}[language=Caml][Objective]
l1 ++ (x, Some y) ++ l2 ++ (y,o) :: tl
\end{lstlisting}
And for every $(u, \text{Some t})$ in $l2$, $t=y$.
\end{lemma}
\begin{proof}
When a vertex $(a,b)$ is added, this is clearly true when $(a,b)=(y,o)$. It is true for the rest of the list (and in all other cases) by induction.
\end{proof}
\begin{lemma}
When $v$ is gray, the stack looks like
\begin{lstlisting}[language=Caml][Objective]
l1 ++ (v, o) :: l2
\end{lstlisting}
Furthermore, $l1$ does not contain $v$ (proved), $l2$ does not contain $(x, \text{Some }v)$ for any $x$ (proved), and if $o=\text{Some }y$, then $l1$ does not contain $y$ as a parent or a child. 
\end{lemma}
\begin{proof}
By induction on dfs-step. The claim is trivially true if we add $(x, None)$ (since the rest of the stack is empty). Thus, assume we are adding $(x, \text{Some }p$). If $(x, \text{Some }p$)= $(v, \text{Some }y)$, then the claim is true, since we do not add $p$ as a child (it must be gray), and the parents are all $p$ ($p\neq x$ since $(x, \text{Some } x)$ never appears in the stack. Otherwise, if $(v, \text{Some }y)$ occurs on the left part of the stack, then we have a contradiction, since $v$ is gray. If it appears on the right, then once again $y$ is not added as a child because it is gray, and it is not added as a parent because 

 When we add a new vertex (trivial in case where we add $(x, None)$), We know that $l1$ does not contain the parent of 
\end{proof}




Known Results:
\begin{lemma}
$\forall$ $x$, $u$, if $(x, Some(u))$ is on the stack, then $u$ is gray.
\end{lemma}
\begin{lemma}
$\forall u$, if $u$ is gray, then $u$ is on the stack.
\end{lemma}
Results to prove:
\begin{lemma}
$\forall x~y$, if $(x, Some y)$ is on the stack, then the stack looks like
\[l1 ++ (x, Some y) ++ l2 ++ (y, o) ++ l3\]
Where every element of $l2$ is of the form $(z, Some y)$ for some $z$.
\end{lemma}
\begin{proof}
By induction on the valid state relation.
\\The base case is trivial because $(x, Some y)$ cannot be on the stack.
\\If we took a step, then we had 3 cases:
\\If the last step was dfs-discover-root, then the stack must be
\[l1 ++ (x, Some y) ++ l2 ++ (y, None) :: nil\]
And by the adding children function (add-to-stack), the claim is true.
\\If the last step was dfs-discover-child, then the stack must be
\[(\text{children of z}) ++ (z, Some u) ++ tl\]
We know that $(x, Some y)$ is on the stack. If it is in the children section, then $y=z$ and the claim is true. If it is $(z, Some u)$ or in $tl$, then it is true by induction, since this part of the stack has not changed.
\\Finally, if the last step was dfs-finish, then the claim is true by induction.
\end{proof}
\begin{lemma}
$\forall s (state), u$, if $u$ is gray, then the stack looks like
\[l1 ++ (u, o) ++ l2\]
For some $o$. We have the following properties:
\begin{enumerate}
\item
$u$ is not in $l1$
\item
$u$ does not appear as a parent in $l2$ (ie, $(x,Some~u)$ is not in $l2$ for any $x$)
\item
If $o=Some~y$, then $y$ does not appear as either a parent or child in $l1$.
\item
If $v$ is in $l1$ and $v$ is gray, then $v$ is a descendant of $u$ in the DFS forest
\end{enumerate}
\end{lemma}
\begin{proof}
We proceed by induction on the claim that $s$ is a valid state. In the start state, this is trivial, since there are no gray vertices.
\\Then, assume it holds for state $s$ and we want to show that it holds for state $s'$ such that $dfs-step~s~s'$ holds.
\begin{enumerate}
\item
If the last step taken was a discover-root, then we know the stack looks like
\[l1 ++ (x, None) :: nil\]
Then, it must be the case that $x=u$ because it is the only gray vertex on the stack. Then $x$ is not in $l1$ (because we do not add the same vertex to the children added to the stack - proved), proving claim $a$. $l2$ is empty, so $b$ is true. $o=None$, so $c$ is vacuously true. Finally, there are no other gray vertices, so $d$ is vacuously true.
\item
If the last step taken was a discover-child, then the stack looks like
\[ \text{(children of x)} ++ (x, Some~y) ++ tl\]
where 
\[(s, Some~y) ++ tl = \text{remove-black } st \]
Where $st$ was the previous stack.
\\Then, we proceed by cases:
\\Case 1: $u=x$. Then $a$ is true by the behavior of adding children. $b$ must be true because if $Some~u$ were in $tl$, then, $u$ would have been gray in the previous state, a contradiction. We know that $y$ must be gray (proved), so it is not added to the children of $x$. Thus, $c$ is true (since all added vertices have a parent $x$). Finally, $d$ is again true vacuously.
\\Case 2: $u\neq x$.
\\Then, the stack must look like
\[ \text{(children of x)} ++ (x, Some~y) ++ l1 ++ (u, o) ++ tl\]
We do not add $u$ to the children of $x$ (since it is gray) and $u\neq u$. $u$ is not in $l1$ by induction, so $a$ is true.
\\$b$ is true by induction (since $tl$ has not changed)
\\
\\Then, we proceed by cases:
\\Case 2.1: $u=y$. In this case, 
\end{enumerate}
\end{proof}
\begin{theorem}
$\forall u, v, G$, if $u\in G$ and $v\in G$, then $d(u) < d(v) < f(v) < f(u)$, $d(v) < d(u) < f(u) < f(v)$, $d(u) < f(u) < d(v) < f(v)$, or $d(v) < f(v) < d(u) < f(u)$.
\end{theorem}
\begin{proof}
It suffices to show that if $d(u) < d(v) < f(u)$, then $f(v) < f(u)$. (We know that $f(v) > d(v)$ already. If the above inequality holds, then at time $d(v)$, it must be the case that $u$ is gray. 
\end{proof}
\end{document}